\subsection {Introduction}
%\addcontentsline{toc}{section}{Introduction}

% GESIS & mission
% WTS & mission
% information extraction at WTS
% RCC
% stage 1 feedback 
% stage 2 approach and results
% remaining chapter

GESIS - the Leibniz Institute for the Social Sciences (GESIS)\footnote{\url{https://www.gesis.org/en/institute}} is the largest European research and infrastructure provider for the social sciences and offers research data, services and infrastructures supporting all stages of the scientific process. The GESIS department \textit{Knowledge Technologies for the Social Sciences (WTS)}\footnote{\url{ https://www.gesis.org/en/institute/departments/knowledge-technologies-for-the-social-sciences/}} is responsible for developing all digital services and research data infrastructures at GESIS and aims at providing integrated access to social sciences data and services. Next to traditional social sciences research data, such as surveys and census data, an emerging focus is to build data infrastructures able to exploit novel forms of social sciences research data, such as large Web crawls and archives. 

Research at WTS\footnote{\url{https://www.gesis.org/en/research/applied-computer-science/labs/wts-research-labs}} addresses areas such as Information Retrieval~(IR), Information Extraction~(IE) {\&} Natural Language Processing~(NLP), semantic technologies and human computer interaction and aims at ensuring access and use of social sciences research data along the FAIR principles, for instance, through interlinking of research data, established vocabularies and knowledge graphs and by facilitating semantic search across distinct platforms and datasets. Due to the increasing importance of Web- and W3C standards as well as Web-based research data platforms, in addition to traditional research data portals, findability and interoperability of research data across the Web constitutes one current challenge. In the context of Web-scale reuse of social sciences resources, the extraction of structured data about scholarly entities such as datasets and methods from unstructured and semi-structured text, as found in scientific publications or resource metadata, is crucial in order to be able to uniquely identify social sciences resources and to understand their inherent relations. 

% prior work: scientifically (disambiguation, extraction, publications, web) etc tools (GESIS datasearch, GWS etc)
Prior works at WTS/GESIS addressing such challenges apply NLP and machine learning techniques to, for instance, extract and disambiguate mentions of datasets\footnote{\url{https://www.gesis.org/en/research/external-funding-projects/archive/infolis-i-and-ii}} \cite{boland2012identifying,ghavimi2016semi}), authors \cite{conf/cikm/Backes18, conf/jcdl/Backes18} or software tools \cite{boland2019distant} from scientific publications or to extract and fuse scholarly data from large-scale Web crawls \cite{journals/semweb/YuGFLRD19, sahoo2015analysing}. Resulting pipelines and data are used to empower scholarly search engines such as the \textit{GESIS-wide search}\footnote{\url{https://search.gesis.org}}  \cite{conf/jcdl/HienertKBZM19} which provides federated search for scholarly resources (datasets, publications etc.) across a range of GESIS information systems or the \textit{GESIS DataSearch} platform\footnote{\url{https://datasearch.gesis.org/}} \cite{Krmer2018ADD}, which enables search across a vast number of social sciences research datasets mined from the Web. 
%TODO other papers worth citing here?

% RCC
Given the strong overlap of our research and development profile with the recent initiatives of the Coleridge Initiative to evolve this research field through the Rich Context Competition (RCC)\footnote{\url{https://coleridgeinitiative.org/richcontextcompetition}}, we are enthusiastic about having participated in the RCC2018 and are looking forward to continue this collaboration towards providing sound frameworks and tools which automate the process of interlinking and retrieving scientific resources.

The central tasks in the RCC are the extraction and disambiguation of mentions of datasets and research methods as well as the classification of scholarly articles into a discrete set of research fields. After the first phase, each team received feedback from the organizers of the RCC consisting of a quantitative and qualitative evaluation. Whereas quantitative results of our inital contribution throughout phase one have shown significant room for improvement, the qualitative assessement, conducted by four judges on a sample of ten documents, underlined the potential of our approach. 

%Judges are then asked to manually extract dataset mentions and calculate the overlap between their dataset extractions and the output of our algorithm.
%Other factors that judges took into consideration are specificity, uniqueness, and multiple occurrences of dataset mentions.
%As for the extraction of research methods and fields, no ground truth has been provided; these tasks were evaluated against the judges' expert knowledge.
%Similarly to the extraction of dataset mentions, specificity and uniqueness have been considered for these two tasks.

Since we have been shortlisted for the second phase of the RCC, this chapter describes our approaches, techniques, and additional data used to address all three tasks. As described in the following subsections, we decided to follow a module-based approach where each module or the entire pipeline can be reused. The remaining chapter is organised as follows.
The following Section~\ref{sec:overview} provides an overview of our approach, used background data and preprocessing steps, whereas Sections ~\ref{sec:dataset-extraction}, ~\ref{sec:research_method_extraction} and ~\ref{sec:field_classification} describes our approaches in more detail, including results towards each of the tasks. Finally, we discuss our results in Section~\ref{sec:discussion} and provide an overview of future work in Section~\ref{sec:conclusion}.
%TODO best to add a short conclusion/future work section.



